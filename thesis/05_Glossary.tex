\section{Glossary}
In the main text every topic is explained before it is used implicityl in the
context. Anyway it is easier to dedicate a section with a small explanation of
every detail.

\textit{}:
\vspace{10pt}

\textit{Buffer}: short way referring to a \textit{Network Buffer}. This is a
packet of 1 MB of data to be sent or received. From external works we know that
this size is the most suitable to maximize Infiniband performances\vspace{10pt}

\textit{CML\_oid}: inside a server, the single packet of data received is split
in CML\_oid, at byte oriented level, and then written to different devices.
\vspace{10pt}

\textit{Device}: in general a device refers to a server's SSD
\vspace{10pt}

\textit{Diagonal Limit}: analytical model to simulate the behaviour of the
network or disks bandwidth. Explained in detail in section \ref{netbuff}
\vspace{10pt}

\textit{Eager Commit}: in the server context, metadata propagation is cached to
send as less network communications as possible. If a client has finished
transmitting a file to a server, it will send the last piece of data as an eager
commit, forcing metadata propagation
\vspace{10pt}

\textit{Geometry}: this measure defined as $D+P$ represents how a parity group
is constructed. $D$ represents data packets and $P$ parity ones. See section
\ref{parity} for more details
\vspace{10pt}

\textit{}:
\vspace{10pt}

\textit{}:
\vspace{10pt}

\textit{Metadata propagation}: in the server context, in order to keep a backup
in case of drive failure, metadata is sent periodically to other servers
\vspace{10pt}

\textit{}:
\vspace{10pt}

\textit{}:
\vspace{10pt}

\textit{}:
\vspace{10pt}

\textit{}:
\vspace{10pt}

\textit{}:
\vspace{10pt}

\textit{Parity Group}: in erasure coding context, a request takes part of a
Parity Group. If a single request is corrupted, the others can recover it based
on the additional information, the parity
\vspace{10pt}

\textit{Parity Map}: given a packet, it has to keep track of its own parity id
and the location of the others packets belonging to the same parity group. This
information is the parity map
\vspace{10pt}

\textit{Parity Request}: after data corruption, the server asks for the packets
with same parity group id of the packets lost to proceed with recovery
\vspace{10pt}

\textit{}:
\vspace{10pt}

\textit{Token}: Before starting a packet transmission, the clients need a token.
The token is consumed upon the transmission start and recovered after the
operation being completed, received the acknowledgment from the server
\vspace{10pt} 

\textit{}:
\vspace{10pt}

\textit{}:
\vspace{10pt}

\textit{}:
\vspace{10pt}


