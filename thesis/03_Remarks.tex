
\section{Development}
Since IME is a complicated environment, has been chosen to develop the simulator 
using an iterative approach. This alloed an understanding of the whole system
showing a single part at a time, a the cost of creating prototypes that needed
to be rewritten partially or completely. \\
To speed up this prototyping approach I've used Dot language \cite{dot-lang} to 
quickly develop some finite state machines that will have been later implemented.

\subsection{Environment choice}

\subsection{Tools Chosen}
\begin{itemize}
    \item Dot graphs
    \item SimPy
    \item PyPy
    \item vmprof \& Flamegraph
\end{itemize}

\subsubsection*{Dot graphs}

\subsection{SimPy Components}
\subsubsection*{Timeout}
The library has some predefined events like a \textit{Timeout} event that just
simulates time passing for the caller.\\
This call blocks only the calling agent, allowing every other event already
triggered to be processed correctly. \\
Every real time consuming action is simulated through this function: the call
with the correct amount of time will make the calling agent to wait until the
task is virtually completed.

\subsubsection*{Resources}
In every simulation environment there are components that needs to be accessed
by many agents and to process correctly this resource contention. This means
keeping a queue of every agent and trigger one as a resource is freed. \\
SimPy takes care of this queue management for us, providing many types of
resources, fitting many needs.
\begin{itemize}
    \item \textbf{Resource} \cite{simpy-resource}: this will keep accounting the quantity
        of agents are accessing this resource at once. It has a free parameter
        \textit{Capacity}, letting the user to set the maximum number of agents
        that can access this resource at once. Setting a capacity = 1 means this
        resource will behave as a mutex, while other values can allow more
        accesses at the same time, based on the real resource behaviour.
    \item \textbf{Container} \cite{simpy-container}: whereas the "Resource" track the
        number of accesses to a resource, the container is concerned about the
        quantity of something stored. As a container cannot store bigger amount
        of goods than its size and cannot give more than it has stored already.
        \\
        A practical example can be an HDD: in the case the disk has already
        200/250 GB of data written, a writing request of 100GB must be blocked
        until the disk has freed some space. \\
        As a side note, the library still lack custom callback implementation to
        handle blocked requests. This means that I don't have the possibility to
        perform some actions if a disk is too full based on SimPy classes. This
        forces me to not fully rely on SimPy classes and implement my own
        custom solutions.
\end{itemize}

\subsection{Tools created}
\begin{itemize}
    \item Monitors
    \item Logger
    \item Tester (Still to be created. The idea is stressing the system with
        multiple tests and get some statistic about the system
\end{itemize}

\subsection{Object size analysis}
Show the size of a set of data
\begin{itemize}
    \item on their own
    \item in a list
    \item in a class
    \item in a queue
\end{itemize}
