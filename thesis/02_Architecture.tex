
\chapter{IME Architecture}
IME system is based on a Client-Server interaction: every compute node has a
Client side process intercepting IO communications that are sent to IME servers.
\begin{myimage}{map}{System representation}
    Graphical layout of an IME node. Clients have a local disk and a connection
    to the IME servers through a Network Fabric. Each server has a set of SSDs
    installed acting as a cache
\end{myimage}
\newpage
In a possible configuration , as shown in figure \ref{eq:map}, can be detected
some different components:
\begin{itemize}
    \item 3 clients with a single HDD each. This hard drive is used to store
        the local image of the OS. In modern data center, client nodes tend
        to be disk-less (they boot over the network). Local HDD can be used
        to store the /tmp and the /var and various Linux system log
        mechanism. But applications are doing most (if not all) of their I/O
        using the fabrics.
    \item a Network Fabric that interconnects every component of IME
    \item up to 20 SSDs per server, that can store data faster compared to the single
        client's HDD
\end{itemize}

Given the skeleton of the architecture, can be added or removed clients,
servers and disks for each one. This changes will end up in different performances
that the simulator must be able to detect. \\
The aim of this simulator is to detect the bottlenecks in order to gain knowledge
on which parts need to be improved allowing the system to scale.
Every configuration and test will be discussed in depth in the analysis section
(\ref{sys-analysis}).

\section{Network communication time}\label{netbuff}
Network communication is one of the main task that are committed in this system,
so an accurate model to represent it is needed. \\
The parameters that determine the time elapsed in a network transaction are
\textit{latency} and \textit{bandwidth}. A file theoretically is sent dividing
its size by the bandwidth available. In the real world we have to consider also
the latency present inside the system: the smallest file possible will be sent
only after $<latency>$ amount of time. \\
\begin{myimage}{ib-bandwidth}{Measured IB bandwidth}
    Infiniband network performs differently based on the requests size involved
    in the transaction. The plot represent different performance based on IB
    architecture and request size. IME has installed EDR Infiniband so the upper
    blue curve represents IME network speed. This data come from a work of
    Jean-Thomas Acquaviva as shown in \cite{erasurecoding-pdf}
\end{myimage}

From figure \ref{eq:ib-bandwidth} we know that the optimal packet size for
communication over infiniband is 1MB. \\
IME Client packs together requests smaller than this threshold and split bigger
files to achieve this communication pattern. This is further referred as
\textbf{Network Buffer}.  If requested explicitly, every request can be flushed
in order to complete a communication in case of remainders.  This set a domain
over the size of the packet sent: we always will have packets that are in the
interval $]0, 1024] KB$. The model should predict correctly the time required by
any network communication inside this interval. \\

\vspace{0.5cm}
\begin{tabular}{c | c}
    file size (KB) & time required \\ \hline
    0 & \textit{latency} \\ \hline
    1024 & \textit{1024KB / bandwidth}
\end{tabular}
\vspace{0.5cm}

These are respectively the worst and the best usage of the network. \\
Having the possibility to choose my interpolating function, I decided to use a
function with a diagonal limit, namely $\lim_{x \to +\infty} y(x) = +\infty$,
that is a real behaviour for every network. \\
The function of choice then is an \textit{hyperbole} adjusted to match the
specified coordinates. \\
\begin{myimage}{diaglimit}{Network transaction time}
    The diagonal limit function applied in the network and disk performance model
\end{myimage}


\section{Tokenized communication}
IME has been designed to be employed with a large number of both clients and
servers. This lead to many parallelization problems and resource management.
With a large amount of clients, the network bandwidth will run out very quickly.
To avoid denial of service or resource stavation, IME is using tokenized
communication, the standard way to implement throttling. The concept of the
token is similar to the Token Ring communication \cite{token-ring} that prevents
a single client to use all the resources of the system and sharing equally the
bandwidth among all the clients.  Here instead of having a single token shared
over all the clients, each client has a fixed amount of tokens.\\
The token usage is the following:
\begin{itemize}
    \item Consumed when making a request
    \item Recovered when receiving an answer
\end{itemize}
This means that for \textit{n} token, a single client can have \textit{n}
communications in parallel.  So far DDN already set a value of 24 tokens that
are not completely used in real applications, meaning that there is not a
bottleneck, but is anyway a free parameter that can be inspected and adapted in
the simulator.



\section{File queuing}
Every clients has internally a write queue for every single server. As it is
asked to write a file, the client will scatter the single file to multiple
queues in order to make us of most servers and speed up the transaction.
When a client wants to perform a write it has to care about:
\begin{itemize}
    \item The amount of servers that will be involved depending on the file size
    \item Not scattering too much a single file to avoid unnecessary broadcast
        operations reading a small amount of data
    \item Perform small transactions of 1MB to maximize network bandwidth
\end{itemize}
Since this system involves not only disk operations but also network ones, we
must be careful to not deplete the network resources. This means that to
maximize an IO operation, make sense using as many devices as possible, but in
a network environment is better to reduce at minimum the packets emitted. \\ To
overcome these opposing behaviour, IME has 2 layers of file queuing:
\textbf{bucket}-based and \textbf{buffer}-based queuing systems.  A bucket is
considered as a part of a file of a fixed size. A single file is split in
bucket-sized parts and then queued to be split again to buffer-sized parts, in
order to perform a network communication.

\begin{myimage}{queuing}{Queueing system using buckets}
    Files to be sent are initially queued. To decide their targets, or
    their parts if splitting is needed, these are queued considering
    chunks bigger than network buffers, called \textit{buckets}. Then every
    bucket is split in network Buffer sized chunks and later sent
\end{myimage}
\begin{itemize}
    \item 4 servers are installed
    \item the user asked to write every file, 32, 12, 16 MB respectively, in
        chronological order.
    \item Bucket size = 8MB
    \item Buffer size = 2MB, just for plotting purposes
    \item parity is not considered
\end{itemize}
In figure \ref{eq:queuing} the first queue is the file-oriented one: every file as it seen
by the client point of view.  Then the central box represents the queue to each
server with the bucket view of the files.  The last one represents also the
server's queues but with network buffers instead of buckets. \\ This different
granularity allows to make us of the server's devices if a file is big enough,
saving at the same time network operations when the user asks to read those
later.\\
\begin{myimage}{queuing-nobuckets}{Queuing system not using buckets}
    If buckets are not used, a file would be more spread over the servers,
    meaning more read requests generated and a higher network usage.
\end{myimage}
The upper image shows how packets should be distributed without the bucket queuing layer.
Note that, from the buffer point of view, the green file could be
split over all the servers, that translates in great write operation but poor
read operation, since it has to perform a broadcast over all the system. This
example must be thought with bigger numbers: in a case with 50 servers, with a
buffer point of view, a file of 50MB involves a broadcast, while the bucket
point of view mitigate this behaviour increasing the file size required by a
broadcast to 400 MB. \\
Some examples in the following table shows the \textit{number of read requests generated}
from a single client reading a file, using the bucket view or not where 50 servers are installed.

\vspace{0.5cm}
\begin{tabular}{l | c | c}
    & \multicolumn{2}{|c}{Number of requests using} \\
    Data read & also buckets & only buffers \\ \hline
    1 MB & 1,2 & 1,2 \\
    8 MB & 1,2 & 8,9 \\
    48 MB& 6,7 & 48,49 \\
    400 MB & 50 & 50 \\
\end{tabular}
\vspace{0.5cm}

For requests size below 400MB there are benefits reducing the number of read
requests generated and so the network traffic. For requests bigger than 400MB
we cannot see the difference since network traffic reduction has been overcome
by IO operations.

\subsection{Queue Issues}
Using the buckets system, targets must be decided in this step and then the
buffer queue adapt the data to be sent at a different granularity.
The parity groups are chosen based on a pseudo-random sequence that should
even out the load to the servers. \\
This system has some issues when some queue is empty: since the target are
not chosen based on the length of the queue, it may happen that some queue is
empty and the parity group will not be complete. \\

\begin{myimage}{packets_choice}{Packets choice based on parity groups}
    Every row represents the queue to a server while every column represents the
    parity map. A circle means that for the creation of that parity group, a
    packet from that queue must be popped and sent.
\end{myimage} \\
In figure \ref{eq:packets_choice} each row is a queue while every column shows a parity group.
A circle shows that an element should be picked from that queue. \\
In the first column the parity group identified by \textit{106} shows that
should be popped an element from queue \textit{1,3,5} and \textit{6}. The
integer identifier is the binary representation of the queues involved. In the
first column $(106)_{10} = (1101010)_2$ \\
In this case the algorithm has a bit of tolerance: at the beginning is more
likely that a single queue does not have any element so, before dropping this
method, it tries some more times to send full parity groups. The chances are in
the order of the server count. \\
After this approach failed more than the chances allowed, another approach is applied:
parity groups are sent based on what is left, meaning that the precomputed
parity groups are not the only groups inside the system, but new ones can be
created based on the needs.\\

%\subsection{HDD behaviour}\label{hdd-behaviour}
%There is a lower minimum size an HDD interact with. If the user wants to edit a
%single byte on the file system, the HDD will read a bigger amount of data,
%accept user modification, and write again the bigger amount of data. If this
%amount of data is 256KB, for instance, performing 8 write of 256KB or 1B each,
%will result in the same performances.  The only difference is that, if we are
%aware of this fact, we can make use of it writing 2MB instead of 8B
%respectively.

\section{Distributed Hash Table (DHT)}
IME servers must be aware of the data that are inside the system, being also
fault tolerant in case of failures. These should also be careful of not flooding
the network with too many requests. \\
DDN used a DHT to solve this problem, allowing the scaling of the number of
servers and allowing the failure of a single one. \\
As a file is stored to a server, its metadata, its way to access it are sent to
another server responsible of delivering the accidentally lost data.

\section{Server side write}
As the server receive a buffer, a number of steps are performed before sending
back an ack to the client. \\ Two separate task are performed, divided in two
different planes: \textbf{Control plane} and \textbf{Data plane}

\subsection{Control plane}
Here is performed metadata propagation if necessary and done some
\textit{logging} operation to a local journal device. \\ Metadata propagation
involve the communication of the current metadata to another machine, meaning
that this operation has to wait an ack before being completed.  This can hit
very hard the performances, forcing an additional network communication for
every write. So this operation is cached instead and done as one of the
following condition is met:
\begin{itemize}
    \item Client requested an eager commit
    \item Cache limit reached. After 128 requests is sent a single request
        instead of 128
    \item Timeout triggered. If no write happened recently, current
        data must anyway be saved so is forced a propagation
\end{itemize}
Metadata propagation is a necessary step in order to recover data in case of
failures: the metadata can be read from the DHT of another working server, then
data can be recovered asking to every involved server the lost parity groups.

\subsection{Data plane}
Data received is stored locally as well as metadata. Since data and metadata
are accessed in a different pattern, and metadata represent a bottleneck when
it comes to files reading, metadata are stored in dedicated devices, less
capable but with wider bandwidth. \\
The problem to face in this case is to make use of every device understanding
the way every device works. \\
As for network communications, the diagonal limit behaviour (see section
\ref{netbuff}) can be applied also in this case. Then we look for bigger
transactions, avoiding small ones. \\
The request is seen in a twofold way: \\

\begin{itemize}
    \item As a set of file parts. These informations are stored as metadata, in
        order to know exactly where each file part is stored. In the picture
        this is represented by the upper layer. In this case the user sent files
        \textit{a,b,c,d,e} and \textit{f}
    \item As a byte stream: aware of the behaviour of HDD (see section
        \ref{hdd-behaviour}) we are now interested in the data as a set of bytes
        to be written to the disk, nothing more. Metadata will tell us the exact
        location of a file, but we know want to perform the most efficient
        operation. \\
        In this step the network is split in \textbf{CML\_oid}s, chunk of 128KB.
        Each of these will be stored to different device, if possible, in a
        round-robin fashion to distribute equally the load. \\
        In the picture the CML\_oid-s generated are the light blue boxes
\end{itemize}

\begin{myimage}{cmloid}{File to CML\_oid conversion}
    The received files are stored as CML\_oid, a byte-level view of the incoming
    data. Every CML\_oid can store a fixed amount of data
\end{myimage}

\section{Data Read}
If a client wants to read a file, or a part of it, it interrogates its file map
to make a request to the interested server, saving network communication
avoiding a broadcast if not necessary. \\
After the server received the request, it has to interrogate its local DHT in
order to know from which devices has the requested data. A read operation is
then performed on those devices and data is sent back to client using as always
a buffered communication of 1 MB per packet. \\
This is the optimal work flow, that is without any kind of error.



%\section{IO operations}
%IME supports different IO operations depending on the task
%\begin{itemize}
    %\item \textbf{Sync}: the content of IME is copied to the mass storage, in
        %order to have 2 copies of it. \\
        %An use case can be keeping a remote file consistent to be read from
        %others server later, leaving the chance to edit it later on IME.
    %\item \textbf{Purge}: the files on IME are moved to the mass storage, not
        %leaving any data on IME server. \\
        %An use case is making some space because the disk is too filled, making
        %space for some other data.
    %\item \textbf{Erase}: files on IME are deleted, not making more copies of
        %them. \\
        %Happens when the data stored are just temporary, so there is no need to
        %propagate further in the system.
%\end{itemize}
%
%These operations will be implemented in the simulator as well in order to better
%classify the models. The objective is to test a model against a set of these
%operations and extracting some measure to classify it.
