
\chapter{Development}
Since IME is a complicated environment, has been chosen to develop the simulator 
using an iterative approach. This allowed an understanding of the whole system
showing a single part at a time, at the cost of creating prototypes that needed
to be rewritten partially or completely. \\
To speed up this prototyping approach has been used the Dot language \cite{dot-lang} to 
quickly develop finite state machines that will have been later implemented.

\section{Environment choice}
Based on the conditions explained in the introduction (see section
\ref{requirements}), \textit{Python} seemed a reasonable choice. Mostly because
of its flexibility to changes, not worrying about small detail like memory
addresses and garbage collection. Since this applications is not computationally
intensive, not involving any kind of floating point operation, the cost of an
interpreted language is not so high. \\
Moreover modern features coming from Python 3.5+ made easier to develop a
relatively big applications as this simulator is, like types annotations and
better generators support.

\section{Tools Chosen}
\begin{itemize}
    \item Dot graphs
    \item SimPy
    \item PyPy
    \item vmprof \& Flamegraph
\end{itemize}


\section{Simpy}\label{simpy}
Simpy is the library the projects is based on. It is a Discrete Simulation
library that fulfill the requirements. Exploiting language mechanics like
generators does not add overhead on the application. It is process-based
meaning that every task can be scheduled as a process, not waiting for others
task to be completed. The library is anyway serial, not creating threads by
itself. Anyway it can be integrated using other libraries, like \textit{threading} (see
\cite{threading}), but so far the simulation achieve reasonable wall times,
there is no need to parallelize it.
Simpy offer a framework in which our routines are executed and tracked. The
status of an agent inside the system is tracked using python generators,
lazy iterators that generate the requested iterator only when needed. Because of
the framework environment we need specific functions that returns a generator
Simpy can take care of and others that keep track of the system status. Simpy
functions can return generators to make the virtual clock going on, create other
virtual processes or wait for other processes to be completed, according to the
situation. The other set of functions are those that implement the system's logic.

\section{SimPy Components}
The philosophy of Simpy led to the development of some ready-to-use components
that ease the process management like mutex and task queues.
Anyway Simpy provides only tools to manage the processes, lacking some
structures to extract usage data to investigate at the end of the simulation.
For this reason some more tools has been developed both in Python, for internal
simulation data extraction, and Bash to automate plotting and file management.

\subsection{Timeout}
The library has some predefined events like a \textit{Timeout} event that just
simulates time passing for the caller.\\
This call blocks only the calling agent, allowing every other event already
triggered to be processed correctly. \\
Every real time consuming action is simulated through this function: the call
with the correct amount of time will make the calling agent to wait until the
task is virtually completed.

\subsection{Resources}
In every simulation environment there are components that needs to be accessed
by many agents and to process correctly this resource contention. This means
keeping a queue of every agent and trigger one as a resource is freed. \\
SimPy takes care of this queue management, providing many types of resources,
fitting many needs.
\begin{itemize}
    \item \textbf{Resource} \cite{simpy-resource}: 
        allows multiple agents to access this tool at the same time. A
        request to this resource is blocked if the quantity of agents using it
        is already at the maximum value. The maximum number of agents
        can be set in its constructor. Setting \texttt{capacity = 1} means this
        resource will behave as a mutex. \\
        An example is accessing a disk to write a file: only a single file can be
        written at once, so the device's resource will have a capacity set to 1.
        The writing process must obtain access to this resource before
        proceeding further. \\
        \begin{myimage}{resource}{Resource request management}
            Request management using simpy.Container. A \textit{Resource} with
            still space accepts a new request (left). A full one rejects the
            request instead (right)
        \end{myimage}
        As shown in Figure \ref{eq:resource} the requests are accepted based on
        the current load of the resource. If there is still room for another
        request, as happens in the left case, the request is accepted. \\
        Otherwise if the request is completely filled, the request is blocked
        until some of the already occupying request free the resource.
    \item \textbf{Container} \cite{simpy-container}: whereas the "Resource"
        tracks the number of accesses to a resource, the container is concerned
        about the quantity of a resource stored. As a container cannot store
        bigger amount of goods than its size and cannot give more than it has
        stored already.  \\
        An example is shown if Figure \ref{eq:container} where different put and
        get requests are performed. Put requests (on the right) are allowed if
        there is enough space, get requests are allowed if there is enough good
        stored. \\
        As a side note, the library still lack custom callback implementation to
        handle blocked requests. This means that there is not the possibility to
        perform some actions if a container is too full, blocking some requests.
        This forces to not fully rely on SimPy classes and implement custom 
        solutions that benefits from these callbacks.
        \begin{myimage}{container}{Container Request management}
            A simpy container with a capacity of 100. On the left the requests
            to put a good in the \textit{Container}, allowed only if there is
            enough space. On the right the get requests, allowed only if there
            is enough resource
            currently stored in the container.
        \end{myimage}
\end{itemize}

\subsection{Processes Management}
Inside the SimPy framework in order to simulate a behaviour a task is seen as a
\textit{process}. This is just a concept: a SimPy simulation runs on a single system process.
After having instanced one or more processes, the caller is able to manage its flow based on
the nature of the situation. It can:
\begin{itemize}
    \item instance one or more processes an proceed with the flow, in case the
        call should not wait for the completion and the task created.
    \item wait for the completion of the single process, in case the caller
        should wait for the completion of the task created
    \item create more processes. Proceed with the flow as soon as \textit{only
        one} finishes its execution
    \item create more processes. Proceed with the flow when \textit{every one}
        finishes its execution
\end{itemize}


\section{Tools created}
Since SimPy provide only simulation utilities, lacking data gathering components
and plotting, these features has been developed aside. Specifically has been designed:
\begin{itemize}
    \item \textbf{Logger}: in order to gather data about the components, every
        action is registered using a shared object. Every step is registered, in
        terms of time spent on that action or in times an action is performed.
        At the end of the simulation the data gathered are printed to external
        files, allowing parsing and plotting using external resources.
    \item \textbf{Testing environment}: to inspect the behaviour of the system
        should be easy to specify many test cases using different
        configurations. For this reason bash scripts and makefiles has been
        developed to manage the simulator at a higher level. At the end of the
        multiple simulations, plots are generated based on the data gathered
        from each run.
\end{itemize}



\section{Profiling}\label{profiling}
Time spending operations are simulates as already specified. The simulator is
not a computationally intensive application but as things start to grow, the
wall time increases. This because of the large amount of data needed to run the
application: a file of 1 GB must be split in buckets, network buffer and
finally in \cmloid of size 128 KB. This result in $1GB / 128KB = 8192 \cmloid$
allocated just for a single gigabyte of data written. Some experiments on C
integration has been conducted but the real problem is data allocation and its
management. The usage of suitable data structures for every need plays a key
role in speeding up the simulator. \\
\textit{Talk about xargs and how memory bound the simulator is}


\section{Alternative Python Implementations}
Looking for performances is reasonable to try different implementation of the
python interpreter. The official one is CPython, but other implementation for
different usage are available. Some of them are
\vspace{0.5cm} \\
\begin{tabular}{r | p{0.8\textwidth}}
    \textbf{Name} & \textbf{Description} \\\hline
    Jython & Implementation that runs on the JVM. Python code is compiled into
    bytecode enabling better performances. Also removes the Global Interpreter Lock
    making use of real threads.\\\hline
    IronPython & Just as a matter of compatibility, a version of Python that
    integrates with Microsoft .NET technologies\\\hline
    Pypy & Distribution with a built-in JIT compiler, overcoming the purely
    interpreted nature of python\\
\end{tabular}
\vspace{0.5cm}

Many other implementations exists, but more for integration into specific
environment, like IronPython does, than achieving a different behaviour of the
interpreter. \\
The choice of Jython has been dropped since it's compatible only with python
2.7. From the beginning the simulator has been developed in Python 3.5 due to
useful features for the development of a big project and some utilities on the
generators.

\subsection{Pypy}
Trying a different implementation, Pypy has been chosen. It provides a custom
version of numpy, but the simulator does not make use of numpy. Many tests
showed that an appropriate container from the package \texttt{collection}
behave better that numpy arrays. The custom numpy installation is the only step
required for Pypy to work.\\
The features advertised by Pypy are:
\begin{itemize}
    \item \textit{Speed} using its JIT compiler. Anyway every time the
        application is called, the compilation phase must be run again. \\
        On their website Pypy team advertise a speed up over 7x compared to
        CPython 2.7. Tests on this specific simulator will follow.
    \item \textit{Less memory usage} using a better custom garbage collector
    \item \textit{Compatibility} so it can run any existing python applications
\end{itemize}

Testing the actual performances of Pypy over CPython I run the simulator with a
reasonable large request pattern: a single client asked to write 8192 files of
16MB. Since Pypy should use also less memory, main memory contention between
different processes should be mitigated. So multiple requests of the same size
have been launched in parallel using \texttt{xargs} command. The simulator is
serial, but more requests can be processes at the same time. For this reason the
usage of \texttt{xargs} is required. \\
The tests has been done using an Intel i5 4690 3.5GHz 4core/4threads with the
graphical session turned off to avoid data fluctuation due to other processes in
the system.

\vspace{0.5cm} 
\begin{tabular}{r | c | c | c | c | c}
    \textbf{Time (s) using CPython 3.5} & & & & &\\
    Processes in parallel & First & Second & Third & Fourth & Total \\\hline
    1 & 41.91 & 41.97 & 42.10 & 42.00 & 167.98 \\
    2 & 43.61 & 43.32 & 43.40 & 43.70 & 87.02 \\
    4 & 52.56 & 54.32 & 54.27 & 54.05 & 54.32 \\
\end{tabular}
\vspace{0.5cm}

\vspace{0.5cm} 
\begin{tabular}{r | c | c | c | c | c}
    \textbf{Time (s) using Pypy 3.5} & & & & \\
    Processes in parallel & First & Second & Third & Fourth & Total \\\hline
    1 & 69.29 & 69.31 & 69.31 & 69.65 & 277.56 \\
    2 & 71.92 & 72.77 & 72.41 & 72.37 & 144.33 \\
    4 & 92.03 & 92.06 & 92.60 & 92.05 & 92.60 \\
\end{tabular}
\vspace{0.5cm}

What is clear from the data gathered is that Pypy is not working for our case.
Further discussion will be done in section \ref{flamegraph}. \\
Using a parallelization of 1, the best times are achieved since there is no main
memory contention. As the number of processes start to grow, more contention
happens but since these times are in parallel, they take less time overall. \\
Measuring this phenomenon a result can be extracted from the data using the formula
$(max(P_4) / min(P_1) - 1) *100$ where $max(P_4)$ is the worst process with a
parallelization of 4 and $min(P_1)$ is the best process with no
parallelization.\\
For the \textit{total} case every process can be expressed using an average
value of the times collected so $total\_time_4 * 4 / total\_time_1$
The results follow

\vspace{0.5cm} 
\begin{tabular}{l | c | c}
    \textbf{Memory contention degradation} & Single & Total \\\hline
    CPython & 29.6\% & 29.3\% \\
    Pypy & 33.6 \% & 33.4\% \\
\end{tabular}
\vspace{0.5cm}

This data shows how Pypy behaves better increasing the parallelization. The
different is not critical, but since Pypy is an alternative implementation,
for this specific application there is no reason to choose Pypy over the default
CPython.

\section{Flamegraph}\label{flamegraph}
\begin{myimage}{profile}{Flamegraph profile data}
    Profiling data using Flamegraph. A vector format can be found at
    \underline{\url{https://drive.google.com/open?id=1JOMhRwSq58nZgOFQMo3JHjH0b6mjI7jh}}
\end{myimage} \\

The profiling task has been accomplished using Flamegraph \cite{flamegraph} a
stack trace analyzer generating an interactive vector graphic to inspect
graphically the application. \\
Figure \ref{eq:profile} represents the flamegraph of the simulator performing a
write of the file requested, a read operation of every file written and a
recovery operation due to a disk failure. \\
To better inspect the result, a text version of the data plotted has been
provided. The following are the 20 most expensive functions that are processed
inside the simulator.

\begin{tabular}{r | l | l}
    \textbf{Time(s)} & \textbf{Source File} & \textbf{Function} \\\hline
    6310 & core.py & run \\
    6016 & core.py & step \\
    4273 & events.py & \_resume \\
    1007 & Server.py & \_\_data\_plane \\
    695 & events.py & \_\_init\_\_ \\
    635 & core.py & schedule \\
    593 & ServerManager.py & perform\_network\_transaction \\
    560 & base.py & \_trigger\_get \\
    557 & ServerManager.py & add\_requests\_to\_clients \\
    533 & DataIndexer.py & write\_packet \\
    525 & Server.py & process\_write\_request \\
    454 & Client.py & add\_write\_request \\
    445 & Client.py & \_\_populate\_bucket\_queue \\
    428 & Client.py & \_\_create\_buffers\_from\_buckets \\
    415 & events.py & \_\_init\_\_ \\
    410 & Client.py & \_\_send\_buffers \\
    403 & base.py & \_\_init\_\_ \\
    373 & Client.py & \_\_send\_write\_request \\
    357 & resource.py & \_\_init\_\_ \\
    357 & Client.py & \_\_prepare\_write\_request \\
\end{tabular} \\

Files with a capital letter are from the simulator, lowercase ones are from
SimPy. These are not all different functions: there is the stack to be
considered, so for example the \textit{run} function includes most of the functions above.
Anyway a non negligible part of the simulation is composed by overhead
introduced by the framework, caused by the resource  and processes management,
aside from their execution. \\
Must be noted that multiple solutions can be proposed leading to the final
results. Some of these solutions in early prototypes were creating many SimPy
processes in order to simulate the environment.  Later this kind of approach has
been dismissed since it was introducing a noticeable overhead inside the
framework. Has been adopted an approach with a single agent that can solve more
tasks on his own, using a single SimPy process.  \\
Moreover the time of many tasks could be computed ahead of time avoiding object
instantiation, but doing so takes the simulator closer to an \textit{Analytical
Simulator} behaviour, that is not the environment of choice. The only components
that can be simulated in such a way are those that do not need to wait for
shared resources and can be processed on their own. \\
So the overhead that SimPy is causing right now has already been reduced to its
minimum.

\section{Object size analysis}
\textit{Is this really necessary?}
Show the size of a set of data
\begin{itemize}
    \item on their own
    \item in a list
    \item in a class
    \item in a queue
\end{itemize}

\section{C integration}
\textit{Talk about that C is not a solution since most of the time is spent
building the structures, reason why Pypy fail on this simulator}
